\documentclass[12pt]{article}
\usepackage{amsfonts, amssymb, amsthm}
\usepackage{amsmath}

\pagestyle{myheadings}
\markright{Simon Kurgan\hfill\today\hfill}

\newtheorem{theorem}{Theorem}[section]
\newtheorem{corollary}{Corollary}[theorem]

\begin{document}

\section{Introduction}

\begin{theorem}
    A sequence is an infinite list of numbers that are indexed by $\mathbb{N}$ or a subset of $\mathbb{N}$. We can often write a sequence in the form $a_1, a_2, ... a_n$
\end{theorem}

\begin{proof}[Example:]
    $(A_n)_{n=0}^\infty$
\end{proof}

\begin{theorem}
    We can define a sequence recursively (recursive definition). The Fibonacci sequence $f_n$ is defined as follows:
\end{theorem}

\begin{proof}[Example:]
    \[
    f_n = 
        \begin{cases}
        f_0 = 1, f_1 = 1 \\
        f_{n+2} = f_{n+1} + f_n \text{ if } n \geq 0.
        \end{cases}
    \]
\end{proof}

\newpage

\begin{theorem}
    For each $n \in \mathbb{N}$, the Fibonacci number $f_{3n}$ is an even natural number.
\end{theorem}

\begin{proof}[Proof:]
    We prove this by induction.

    Base case: when $n = 0, f_0$ is even.

    Inductive step: Suppose $f_{3k}$ is even for some $k \geq 0$

    We want to prove $f_{3(k+1)}$ is even

    Note $f_{3(k+1)}$ = $f_{3k+3}$

    By the recursive definition, $f_{3k+3}$ = $f_{3k+2}$ + $f_{3k+1}$

    Further simplifying, ($f_{3k+1}$ + $f_{3k}$) + $f_{3k+1}$

    $2f_{3k+1} + f_{3k}$

    Substituting, $2f_{3k+1} + 2L$

    Thus, $2(f_{3k+1} + L)$ is even.

    By the closure of the set of integers and the recursive definition, this is an integer.

    By induction, this statement is true for any $\mathbb{N}$ 
\end{proof}

\newpage

\begin{theorem}
    How many ways can you tile a 2 by n grid with dominoes?
\end{theorem}

\begin{proof}[Illustrated:]

    Working from a simpler case, suppose n = 1. There is only one way to fill the grid.

    When n = 2, there are only two ways like such $\|$ and $=$

    When n = 3, there are 3 ways in which you can tile the dominoes

    When n = 4, there are 5 ways.

    When n = 5, there are 8 ways.
\end{proof}

\begin{proof}[Illustrated:]
    For any integer $n \geq 1$, the number of ways to tile a 2 by n grid with dominoes is the (n+1)th Fibonacci number, $f_{n+1}$

    Recall,

    \[
    f_n = 
        \begin{cases}
        f_0 = 1, f_1 = 1 \\
        f_{n+2} = f_{n+1} + f_n \text{ if } n \geq 0.
        \end{cases}
    \]

    Proof using induction.

    Base Case: 
    
    Suppose n = 1. There is 1 way to tile a 2 by 1 grid and $f_1 = 1$

    Suppose n = 2. There are 2 ways to tile a 2 by 2 grid and $f_3 = 2$

    Inductive Case: 
    
    Suppose the number of ways to tile an n by k grid is $f_{k + 1}$

    Suppose the number of ways to tile a 2 by (k + 1) grid is $f_{k + 2}$

    Goal: Find out the number of ways to tile a 2 by (k + 2) grid.

    Consider the top left square of this 2 by (k + 2) grid.

    There are only two ways in which it can be covered

    Case 1: 

    This square is covered by a vertical d.

    The remaining part is a 2 by (k + 1) grid.

    By the inductive hypothesis, there are $f_{k + 2}$ to cover the grid.

    Case 2:

    The square is covered by a horizontal d.

    The square underneath it must be covered by a horizontal domino.

    The remaining grid is a 2 by k grid. Which has $f_{k + 1}$ ways to tile.
\end{proof}

\newpage

\end{document}
