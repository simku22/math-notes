\documentclass[12pt]{article}
\usepackage{amsfonts, amssymb, amsthm}
\usepackage{amsmath}

\pagestyle{myheadings}
\markright{Simon Kurgan\hfill\today\hfill}

\newtheorem{theorem}{Theorem}[section]
\newtheorem{corollary}{Corollary}[theorem]

\begin{document}

\section{Introduction}


\begin{theorem}
    
\end{theorem}

\begin{proof}[Proof:]
    We prove this by induction.

    Base case: when $n = 0, f_0$ is even.

    Inductive step: Suppose $f_{3k}$ is even for some $k \geq 0$

    We want to prove $f_{3(k+1)}$ is even

    Note $f_{3(k+1)}$ = $f_{3k+3}$

    By the recursive definition, $f_{3k+3}$ = $f_{3k+2}$ + $f_{3k+1}$

    Further simplifying, ($f_{3k+1}$ + $f_{3k}$) + $f_{3k+1}$

    $2f_{3k+1} + f_{3k}$

    Substituting, $2f_{3k+1} + 2L$

    Thus, $2(f_{3k+1} + L)$ is even.

    By the closure of the set of integers and the recursive definition, this is an integer.

    By induction, this statement is true for any $\mathbb{N}$ 
\end{proof}

\newpage

\begin{theorem}
    How many ways can you tile a 2 by n grid with dominoes?
\end{theorem}

\begin{proof}[Illustrated:]

\end{proof}

\begin{proof}[Illustrated:]

\end{proof}

\newpage

\end{document}
