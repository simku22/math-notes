\documentclass[12pt]{article}
\usepackage{amsfonts, amssymb, amsthm}
\usepackage{amsmath}
\usepackage{amsthm}

\pagestyle{myheadings}
\markright{Simon Kurgan\hfill Homework 4 \hfill}

\newtheorem{theorem}{Problem}[section]
\newtheorem{corollary}{Corollary}[theorem]

\begin{document}

\section{Assigned Problems}

\begin{theorem}
    For every $a, b, c \in \mathbb{R}$, if $a < b$ and $c > 0$, then $ac < bc$.
\end{theorem}

\begin{proof}[Proof:]
    Suppose $a < b$ and $c > 0$.

    If $b - a$ is positive, and $c$ is positive by Theorem 6

    By positive closure, $(b-a)(c) \in \mathbb{R}$

    If $ac < bc$, then $bc - ac > 0$
    
    By distributivity, $bc - ac$ = $c(b-a)$

    By closure, $c(b-a) > 0$ thus satisfying $ac < bc$



    % \begin{align*}
    %     4xyzw &= 2\cdot2tu \\
    %     &\le 2\cdot(t^2+u^2) \\
    %     &= 2\cdot((xy)^2+(zw)^2) &&\text{(substituting variables)} \\
    %     &= 2\cdot(x^2y^2+z^2w^2) \\ 
    %     &= 2x^2y^2+2z^2w^2 \\
    %     &\le ((x^2)^2+(y^2)^2)+((z^2)^2)+(w^2)^2) \\
    %     &= x^4+y^4+z^4+w^4 &&\qedhere
    % \end{align*}

    % So, for $a = -5$ and $b = 2$, there exists a unique $q$ and $r$ belonging to the set of integers such that $a = 2q + r$ with $0 \leq r < 2$ (i.e., $r = 0$ or $1$).
    
    % By Theorem 16, every integer is either odd or even.
\end{proof}

\newpage

\begin{theorem}
For every $a, b \in \mathbb{R}$, $ab > 0$ if and only if $a$ and $b$ are both positive or both negative.
\end{theorem}

\begin{proof}[Proof:]

    \textbf{$\longrightarrow$}

    Suppose $ab > 0$
    
    Goal: $a, b \in \mathbb{R^+} \lor a, b \in \mathbb{R^-}$

    \textbf{Case 1:} $a, b \in \mathbb{R^+}$

    By closure of positive real numbers, $a * b \in \mathbb{R^+}$
    
    By theorem 6, such $(a * b) > 0$

    \textbf{Case 2:} $a, b \in \mathbb{R^-}$
    
    QUESTION: If we prove case 1, why prove case 2? 

    \textbf{$\longleftarrow$}

    Suppose $a, b \in \mathbb{R^+} \lor a, b \in \mathbb{R^-}$

    By conditional laws, 
    
    $(A \lor B) \implies C$ 
    
    $\equiv C \lor \lnot (A \land B)$ 

    $\equiv C \lor (\lnot A \land \lnot B)$
    
    $\equiv (C \lor \lnot A) \land (C \lor \lnot B)$
    
    $(A \implies C) \land (B \implies C)$

    % https://math.stackexchange.com/questions/4637016/how-to-prove-a-statement-of-a-or-b-implies-c

\end{proof}

\newpage

\begin{theorem}
For every integer $n$, $n^3 \equiv n \pmod{3}$.
\end{theorem}

\begin{proof}[Proof:]
By Corollary 1, we only have one of the three cases:

\textbf{Case 1:} $n \equiv 0 \pmod{3}$

\[
n^3 \equiv n \pmod{3} \longrightarrow  3 \mid (n^3 - n)
\]
\[
3 \mid (n^3 - n) \longrightarrow (n)(n - 1)(n - 1)
\]
By Theorem 20, we have $n^3 \equiv 0^3 \pmod{3}$.

\textbf{Case 2:} $n \equiv 1 \pmod{3}$

Similarly, $n = n^3 \pmod{3}$.

\textbf{Case 3:} $n \equiv 2 \pmod{3}$

\end{proof}

\newpage

\begin{theorem}
Let $n \in \mathbb{Z}$. Then $n$ is even if and only if $n^2$ is even. Similarly, $n$ is odd if and only if $n^2$ is odd.
\end{theorem}

\begin{proof}[Proof:]
\textbf{$\longrightarrow$}

Suppose $n$ is even. Goal: $n^2$ is even.

Then $n = 2k$ for some $k \in \mathbb{Z}$.

So, $n^2 = (2k) \times n = 2(kn)$, where $kn \in \mathbb{Z}$ by the closure of $\mathbb{Z}$.

Hence, $n^2$ is even.

\textbf{$\longleftarrow$}

Need to prove: if $n^2$ is even, then $n$ is even.

Contrapositive: If $n$ is not even, then $n^2$ is not even.

By Theorem 16, if an integer is not even, then it is odd.

Suppose $n$ is odd. Then $n = 2k + 1$ for some $k \in \mathbb{Z}$.

\[
\text{So, } n^2 = (2k + 1)^2 = 4k^2 + 4k + 1 = 2(2k^2 + 2k) + 1.
\]

Since $2k^2 + 2k \in \mathbb{Z}$, $n^2$ is odd.
\end{proof}

\newpage

\section{Revisit}

Let $x \in \mathbb{R}$. We say $x$ is rational if there exist integers $p$ and $q$ with $q \neq 0$ such that $x = \frac{p}{q}$. We say $x$ is irrational if it is not rational.

\begin{theorem}
$\sqrt{2}$ is irrational.
\end{theorem}

\begin{proof}[Proof:]
There exist $p, q \in \mathbb{Z}$ such that $\sqrt{2} = \frac{p}{q}$, $q \neq 0$.

$2 = \frac{p^2}{q^2} \Rightarrow 2q^2 = p^2$

$\Rightarrow p^2$ is even

$\Rightarrow p$ is even ($p = 2k$, where $k \in \mathbb{Z}$)

$\Rightarrow 2q^2 = (2k)^2 = 4k^2$.

$q^2 = 2k^2$

$\Rightarrow q^2$ is even

$\Rightarrow q$ is even ($q = 2L$, where $L \in \mathbb{Z}$)

$\sqrt{2} = \frac{p}{q}$ p is negative and $-p$ is positive.
\end{proof}

\textbf{Remark:} We use proof by contradiction, then we prove that the well-ordering property is contradicted.

\end{document}
