\documentclass[12pt]{article}
\usepackage{amsfonts, amssymb, amsthm}
\usepackage{amsmath}
\usepackage{amsthm}
\usepackage{array}

\pagestyle{myheadings}
\markright{Simon Kurgan\hfill Homework 4 \hfill}

\newtheorem{theorem}{Problem}[section]
\newtheorem{corollary}{Corollary}[theorem]

\begin{document}

\section{Assigned Problems}

\begin{theorem}
    For every $a, b, c \in \mathbb{R}$, if $a < b$ and $c > 0$, then $ac < bc$.
\end{theorem}

\begin{proof}[Proof:]
    Suppose $a < b$ and $c > 0$.

    By order axioms $b - a$ is positive, and $c$ is positive by Theorem 6

    By closure, $(b-a)(c) \in \mathbb{R^+}$

    If $ac < bc$, then by order axioms we can write $bc - ac > 0$
    
    By distributivity, $bc - ac$ = $c(b-a)$

    By commutativity, $c * (b-a) = (b-a) * c$

    By above closure, $c * (b-a) > 0$ thus satisfying $ac < bc$



    % \begin{align*}
    %     4xyzw &= 2\cdot2tu \\
    %     &\le 2\cdot(t^2+u^2) \\
    %     &= 2\cdot((xy)^2+(zw)^2) &&\text{(substituting variables)} \\
    %     &= 2\cdot(x^2y^2+z^2w^2) \\ 
    %     &= 2x^2y^2+2z^2w^2 \\
    %     &\le ((x^2)^2+(y^2)^2)+((z^2)^2)+(w^2)^2) \\
    %     &= x^4+y^4+z^4+w^4 &&\qedhere
    % \end{align*}

    % So, for $a = -5$ and $b = 2$, there exists a unique $q$ and $r$ belonging to the set of integers such that $a = 2q + r$ with $0 \leq r < 2$ (i.e., $r = 0$ or $1$).
    
    % By Theorem 16, every integer is either odd or even.
\end{proof}

\newpage

\begin{theorem}
For every $a, b \in \mathbb{R}$, $ab > 0$ if and only if $a$ and $b$ are both positive or both negative.
\end{theorem}

\begin{proof}[Proof:]

    \textbf{$\longrightarrow$}

    Suppose $ab > 0$
    
    Goal: $a, b \in \mathbb{R^+} \lor a, b \in \mathbb{R^-}$

    By the trichotomy of real numbers, there exist 3 possibilities for both $a, b$

    Thus, $3 * 3 = 9$ total combinations

    \begin{center}
        \begin{tabular}{|c|c|c|}
            \hline
            \textbf{Case \#} & a & b \\
            \hline
            1 & $a > 0$ & $b > 0$ \\ 
            2 & $a > 0$ & $b < 0$ \\  
            3 & $a > 0$ & $b = 0$ \\  
            4 & $a < 0$ & $b > 0$ \\ 
            5 & $a < 0$ & $b < 0$ \\ 
            6 & $a < 0$ & $b = 0$ \\ 
            7 & $a = 0$ & $b > 0$ \\ 
            8 & $a = 0$ & $b < 0$ \\  
            9 & $a = 0$ & $b = 0$ \\  
            \hline
        \end{tabular}
    \end{center}

    Cases 3, 6, 7, 8 where either $(a = 0)$ or $(b = 0)$ by Theorem 4 equate to 0, because some $a * 0 = 0$ where $a \in \mathbb{R}$.
    
    This contradicts $a * b > 0$ because $a * b = 0$ in all 4 cases.

    Similarly, case 9 where both $a, b = 0$

    By theorem 4, $a * 0 = 0$, thus $0 * 0 = 0$

    By substitution, we can rewrite $0 * 0 = 0$ as $a * b = 0$

    This contradicts $a * b > 0$ and the trichotomy principle.

    Thus remains, 

    \begin{center}
        \begin{tabular}{|c|c|c|}
            \hline
            \textbf{Case \#} & a & b \\
            \hline
            1 & $a > 0$ & $b > 0$ \\ 
            2 & $a > 0$ & $b < 0$ \\  
            4 & $a < 0$ & $b > 0$ \\ 
            5 & $a < 0$ & $b < 0$ \\ 
            \hline
        \end{tabular}
    \end{center}

    \newpage

    For case 1, suppose both $a, b > 0$

    By theorem 6, both a, b are positive.

    By positive closure, $a * b \in \mathbb{R^+}$, by definition $a * b$ is positive and $a * b > 0$

    Our result is consistent with $ab > 0$

    For case 2, suppose $a > 0$ and $b < 0$

    By theorem 6, a is positive.

    By order axioms and negativity, $b < 0 \equiv -b > 0$ 

    By positive closure, $-b * a \in \mathbb{R^+}$.

    By multiplication, $-b * a \equiv -ba$.

    By order axioms and the definition of positivity, if $-ba \in \mathbb{R^+}$ then $-(-ba) \in \mathbb{R^-}$.

    By E7, $-(-ba) = ba$

    By commutativity, $ba = ab$

    Thus, we derive $ab \in \mathbb{R^-}$, but this contradicts our hypothesis.

    For case 4, suppose $a < 0$ and $b > 0$

    By theorem 6, a is positive.

    By order axioms, $a < 0 \equiv -a > 0$

    By positive closure, $-a * b \in \mathbb{R}$

    By multiplication, $-a * b = -ab$

    By order axioms and the definition of positivity, if $-ab \in \mathbb{R^+}$ then $-(-ab) \in \mathbb{R^-}$.

    By E7, $-(-ab) = ab$

    Thus, we derive $ab \in \mathbb{R^-}$, but this contradicts our hypothesis.

    The final case 5, in which $a, b < 0$

    By positivity and order axioms $a < 0$ then $-a > 0$ and similarly $b < 0$ then $-b > 0$.

    By positive closure, $-a * -b \in \mathbb{R^+}$.

    By theorem e11, $-a * -b = ab$

    Thus we have $ab \in \mathbb{R^+}$.

    The only affirming cases here are 1, 5 where $a,b \in \mathbb{R^+}$ or $a, b \in \mathbb{R^+}$

    Thus satisfying this statement.

    \textbf{$\longleftarrow$}

    Suppose $a, b \in \mathbb{R^+} \lor a, b \in \mathbb{R^-}$

    Case 1: $a, b \in \mathbb{R^+}$

    By positive closure, $a * b \in \mathbb{R^+}$

    By theorem 6 and positivity, $a * b > 0$

    \newpage

    Case 2: $a, b \in \mathbb{R^-}$

    By negativity and order axioms if $a < 0$ then $-a > 0$ and similarly if $b < 0$ then $-b > 0$.

    By positive closure, $-a * -b \in \mathbb{R^+}$.

    By theorem E11, $-a * -b = ab$

    By positivity and theorem 6, $ab > 0$

\end{proof}

\end{document}
