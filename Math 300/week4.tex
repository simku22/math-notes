\documentclass[12 pt]{article}        	%sets the font to 12 pt and says this is an article (as opposed to book or other documents)
\usepackage{amsfonts, amssymb}	
\usepackage{amsmath}				% packages to get the fonts, symbols used in most math


\pagestyle{myheadings}                           	% tells LaTeX to allow you to enter information in the heading
\markright{Simon Kurgan\hfill \today \hfill} 	% put your name instead of Murphy Waggoner 
																									% and put the proposition number from the book
                                                	% LaTeX will put your name on the left, the date the paper 
                                                	% is generated in the middle 
                                                 	% and a page number on the right



\newcommand{\eqn}[0]{\begin{array}{rcl}}%begin an aligned equation - allows for aligning = or inequalities.  Always use with $$ $$
\newcommand{\eqnend}[0]{\end{array} }  	%end the aligned equation

\newcommand{\qed}[0]{$\square$}        	% make an unfilled square the default for ending a proof

%\doublespacing                         	% Together with the package setspace above allows for doublespacing of the document

\begin{document}												% end of preamble and beginning of text that will be printed

        																% makes the word Proposition and the proposition number bold face  
% \textbf{Proposition R.231:}							% the Proposition number from the book (this one is fictitious)
% Prove that $A = \left\{m + n\sqrt{3}\ |\ m,n \in \mathbb{Z} \right\}$ is closed under mulitplication.
                                   
% 																				% be sure to leave at least one blank line here so that 
% 																				% the Proof starts with a new paragraph

% \textbf{Proof:}              						% makes the word Proof bold face
% Let $A = \left\{m + n\sqrt{3}\ |\ m,n \in \mathbb{Z} \right\},$ 
% and let $m + n\sqrt{3}$ and $p + q\sqrt{3}$ be elements of $A$.
% Then 
% $$\eqn 
% \left( m + n\sqrt{3} \right)\left(p + q\sqrt{3} \right) & = & mp + mq\sqrt{3} + np\sqrt{3}  + 3qn\\
% 																												& = & (mp + 3qn) + (mq + np)\sqrt{3}.\\    
% 																															  \eqnend$$
% Since $m, n, p, q \in \mathbb{Z}$, $mp + 3nq$ and $ mq + np$ are both integers.  Therefore, 
% $$\left( m + n\sqrt{3} \right)\left(p + q\sqrt{3}   \right) \in A,$$ 
% and $A$ is closed under multiplication. \qed


\textbf{Theorem 21}							% the Proposition number from the book (this one is fictitious)
For all integers $a$ and $b$ with $b > 0$, there exist unique integers $q$ and $r$ such that $a = bq + r$, and $0 \leq r < b$.

\textbf{Proof:}   
Let $a = -5$ and $b = 2$. Then,

\[
-5 = 2 \times (-3) + 1
\]

So, for $a = -5$ and $b = 2$, there exists a unique $q$ and $r$ belonging to the set of integers such that $a = 2q + r$ with $0 \leq r < 2$ (i.e., $r = 0$ or $1$).

By, theorem 16, every integer is either odd or even.

\textbf{Corollary 1} 
For any integer a, b with $b > 1$, there exists a unique r in Z.

Such that a = r (mod b) and $0 \leq r < be$

When $b = 3$, for any $z \in \mathbb{Z}$, there exists a unique integer $r$ such that $a \equiv r \pmod{3}$ and $r \in \{0, 1, 2\}$.

\textbf{Theorem 21}	
For every integer n, $n^3 = n (mod 3)$

\textbf{Proof:}
By corollary 1, we only have one of the three cases

Case 1: n = 0 (mod 3)

\[
 	n^3 \equiv n \pmod{3} \longrightarrow  3 \mid (n^3 - n)
\]

\[
	3 \mid (n^3 - n) \longrightarrow (n)(n - 1)(n - 1)
\]

By theorem 20, we will have $n^3 = 0^3 \pmod{3}$


Remark: By reflexivity, $a \equiv b \!\pmod{n} \longrightarrow b \equiv a \pmod{n}$.

Case 2: n = 1 (mod 3)

$n = 1 (mod 3)$, By theorem 20, $n^3 = 1^3 (mod 3)$

Similarly, $n = n^3(mod 3)$

Case 3: n = 2 (mod 3)

\textbf{Theorem 19}	
Let n belong to Z, Then n is even if and only if $n^2$ is even. Similarly, n is odd if and only if $n^2$ is odd.

\textbf{Proof:}

$\longrightarrow$

Suppose, n is even. Goal: $n^2$ is even.

Then n = 2k for some k in Z.

So, $n^2 = (2k) * n = 2(kn)$, where kn is in Z by the closure of Z

Hence $n^2$ is even.

$\longleftarrow$

Need to prove: if $n^2$ is even, then n is even.

$(P \implies Q) \equiv (\lnot Q \implies \lnot P)$

Contrapositive: 

If n is not even, then $n^2$ is not even.

By theorem 16. If an integer is not even, then it is odd.

Suppose n is odd. THen n = 2k + 1 for some k in Z.
\[
\text{So } n^2 = (2k + 1)^2 = 4k^2 + 4k + 1 = 2(2k^2 + 2k) + 1.
\]

\[
\text{Since } 2k^2 + 2k \in \mathbb{Z}, \text{ } n^2 \text{ is odd.}
\]

\textbf{Revisit}
Let $x \in \mathbb{R}$. We say $x$ is rational if there exist integers $p$ and $q$ with $q \neq 0$ such that $x = \frac{p}{q}$. We say $x$ is irrational if it is not rational.

\textbf{Theorem 17:}
$\sqrt{2}$ is not rational.

\textbf{Proof:}
There exist $p, q \in \mathbb{Z}$ such that $\sqrt{2} = \frac{p}{q}$, $q \neq 0$.

$2 = \frac{p^2}{q^2} \Rightarrow 2q^2 = p^2$

$\Rightarrow p^2$ is even

$\Rightarrow p$ is even ($p = 2k$, where $k \in \mathbb{Z}$)

$\Rightarrow 2q^2 = (2k)^2 = 4k^2$.

$q^2 = 2k^2$

$\Rightarrow q^2$ is even

$\Rightarrow q$ is even ($q = 2L$, where $L \in \mathbb{Z}$)

$\sqrt{2} = \frac{p}{q}$ p is negative and -p is positive.

\textbf{Rewriting:}

Assume p is positive (if -p negative, we have $\sqrt{2} = \frac{-p}{-q}$ so we choose -p)

Consider $S = \{ p \in \mathbb{Z}^+ | \sqrt{2} = \frac{p}{q} \text{ for } q \in \mathbb{Z}, q \neq 0 \}$.

Note $S = \varnothing$. Then by the well-ordering property. 

$S$ has a smallest element $p_0$ such that $\sqrt{2} = \frac{p_0}{q_0}$ for some $q_0 \in \mathbb{Z}$, $q_0 \neq 0$, where $p_0$ is $p$ not $p$ subscript $0$.

Then $2 = \frac{p_0^2}{q_0^2}$ so $2q_0^2 = p_0^2$, which implies $p_0^2$ is even.

By Theorem 19, $p_0$ is even, so $p_0 = 2k$ for some $k \in \mathbb{Z}$.

$\Rightarrow 0 < k < p_0$

Plug it into our original equation to have:

$2q_0^2 = (2k)^2 = 4k^2 \Rightarrow q_0^2 = 2k^2$

So, $q_0^2$ is even. By Theorem 19, $q_0$ is even.

This implies $q_0 = 2L$ for some $L \in \mathbb{Z}$

Then $\sqrt{2} = \frac{p_0}{q_0} = \frac{2k}{2l} = \frac{k}{l} \Rightarrow k \in S$.

Since $k < p_0$, we have a contradiction.

Hence, $\sqrt{2}$ is not rational.

$\Box$

\textbf{Remark:} We use proof by contradiction, then we prove that the well ordering property is contradicted

\end{document}